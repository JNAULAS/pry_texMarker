\documentclass[11pt]{article}
\usepackage{graphicx}
\usepackage{color}
\usepackage{enumitem}
% Parquete para matematicas.
\usepackage{amsmath, amsfonts}
% Permte crear referencias
\usepackage{hyperref}
% Corrige idioma de las �
\usepackage[T1]{fontenc}
%\usepackage[spanish]{babel}
\usepackage[latin1]{inputenc}
% Agregamos paquete que me permite citar
\usepackage{cite}

\title{\color{red} Estructura del documento en \LaTeX}
\author{Juan Naula}
\date{}

\begin{document}
\maketitle
\begin{abstract}
    La clase utilizada en esta actividad es \verb|article| y el tama�o de la letra es de 11 puntos. Debe definirse un documento \verb|pdf| que sea id�ntico a este documento.
    Para la elaboraci�n de listas, puede que necesites consultar la referencia proporcionada en al secci�n \textbf{entornos}.
\end{abstract}
% Agregamos la seccion , agregamos indetificador de la seccion \label{sec:1}
\section{Introducci�n} \label{sec:1}
En la secci�n \ref{sec:1} se realiza una practica de listas.
% Agregamos subsecciones
\subsection{Lista}
\begin{itemize}
    \item Primer Elemento
    \begin{itemize}
        \item Primer subelemento
        \item Segundo subelemento
    \end{itemize}
    \item Segundo Elemento
    \item Tercer Elemento
    %Agregamos punto seguidos
    \item \ldots

    % Agregamos nueva seccion de listas enumeradas
    \subsection{Listas enumeradas} \label{sub1}
    \begin{enumerate}
        \item Primer Item
            \begin{enumerate}
                \item Primer subelemento enumerado
                \item Segundo subelemento enumerado
                \item Tercero subelemento enumerado
            \end{enumerate}
        \item Segundo Item
        \item Tercero  Item
    \end{enumerate}
\end{itemize}
\subsection{Descripci�n del elemento} \label{sub2}
\begin{description}
    \item[Rojo] Color que caracteriza el peligro. 
    \item[Azul] Color del cielo 
\end{description}
% Formulas matematicas
\section{Matem�ticas} \label{sec:2}
% Agregamos mod matematico $$ Eline con esto entiende latex
% Latex en modo matematico no respeta texto
% si se requiere poner testo dentro de l aformual se utiliza \textahsjah
En la secci�n \ref{sec:2} realizaremos ecuaciones matem�ticas. La formula de la teroria de la realtividad es $E=mc?2$

\subsection{Exponentes e Indices}
% No debe existir espacion en el equation
\begin{equation}
    %Agregamso subindices _ Permite indicar que es un subindice
    k_{n+1}=n?2+k_n?2-k_{n-1}
\end{equation}\label{EC1}
%Para referenciar la ecuacio
En \ref{EC1} se observa el uso de exponentes e �ndices.


%Subseccion raiz cuadrada par esto se usa la fraccion 
\subsection{Raices Cuadradas} \label{EC2}
\begin{equation}
    \sqrt{\frac{a}{b}}
\end{equation}

%Ecuaciones mas complejas
\begin{equation}
    \sqrt{[n]{1+x+x?2+x?3+\dots}}
\end{equation}

% para no usar begin y usar modo matematico
\[
\frac{n!}{k!(n-k!)}
\]
% Para enumerara debemos usar si o si el section
\[
\frac{\frak{1}{x}+\frac{1}{y}}{y-z}
\]
% Tener cuidado al escribir las letras como corresponden
\subsection{Letras Griegas}
\begin{equation}
    \alpha\mu_1+\beta \mu_2=v
\end{equation}
% Escribri ecuacion inline $n(n+1)/2$
La suma de los $n$ primeros n�meros enteros positivos es $\frac{n(n+1)}{2}$ es decir $1+2+\dots+n(n+1)$.
Utilizando la anotaci�n sumatoria, lo anterior se escribiria como $\sum_{i+1}?{n}=\frac{n(n+1)}{2}$; formula que volveremos a escribir en modo resaltado.

\[
\sum_{i=1}?{n}=\frac{n(n+1)}{2}
\]

% Subsection de matris, llena de puntos de la columna 2
\subsection{Matriz}
\[
\begin{matrix}
    a_{11}&a_{12}&a_{13}&\dots&a_{1n}\\
    a_{21}&a_{22}&a_{23}&\dots&a_{1n}\\
    \hdotsfor[2]{5}\\
    a_{n1}&a_{n2}&a_{n3}&\dots&a_{nn}\
\end{matrix}
\]
% Formas de escribir las ecuaciones
\begin{equation}
    \begin{matrix}
        a&b\\
        c&d
    \end{matrix}
\end{equation}

\begin{equation}
    \begin{pmatrix}
        a&b\\
        c&d
    \end{pmatrix}
\end{equation}

\begin{equation}
    \begin{bmatrix}
        a&b\\
        c&d
    \end{bmatrix}
\end{equation}

\begin{equation}
    \begin{Bmatrix}
        a&b\\
        c&d
    \end{Bmatrix}
\end{equation}

\begin{equation}
    \begin{vmatrix}
        a&b\\
        c&d
    \end{vmatrix}
\end{equation}

\begin{equation}
    \begin{Vmatrix}
        a&b\\
        c&d
    \end{Vmatrix}
\end{equation}
% Ecuaciones mas complejas
\begin{equation}
    \begin{pmatrix}
        1&0\cdots&0\\
        0&1\cdots&0\\
        \vdots & \vdots & \ddots&\vdots\\
        0&0&\cdots&1
    \end{pmatrix}
\end{equation}
% Creacion de casos en los libros
\subsection{Casos}
\[
\operatorname{sing}x=
\begin{cases}
    1,&x>0\\
    0,&x=0\\
    -1,&x<0
\end{cases}
\]
% Agregamso nueva subseccion con ecuaciones 
\subsection{Ecuaciones de varias lineas}
% Realizaremos array de ecuaciones
\begin{eqnarray}
    (a+b)^2-(a-b)^2 &=& \\
    (a+b) (a+b)-(a-b)(a-b) &=& \\
    (a^2+2ab+b^2)-(a^2-2ab+b^2) &=& 4ab
%Para que no salga enumerada se agreag el * en el cierre de la      % ecuacion
\end{eqnarray}
%Anpesan da saldo de columna 
\begin{align}
    (a+b)^2-(a-b)^2&=(a+b)(a+b)-(a-b)(a-b)\\
                   &=(a^2+2ab+b^2)-(a^2-2ab+b^2)\\
                   &=4ab
\end{align}

%Nueva Seccion, latex ordena ups
\section{Figuras}
% Aqui nosotros podemos decirle donde quiero que se ubique al figura %[h], forza la figura a estar donde estaba [h!]
%Aqui debemos dar al ruta de la figura\textbf{}
%Permite dar un nombre
\begin{figure}[h]   
    \centering
    \includegraphics[width=40mm,height=40mm]{logo.png}
    \caption{Logo UPS}
    \label{F1}
\end{figure}

En la Fig. \ref{F1} se muestra el Logo de la Universidad
%Agregar imagen con acnho de columnas
%Aqui administra el espacio evitando que se deforme y con esto se %evita estar cuadrando la imagen
%Aqui debemos dar al ruta de la figura
%\includegraphics[scale=1.2, angle=45]{logo.png}
\begin{figure}[h]
    \centering
    \includegraphics[width=0.3\textwidth, angle=180]{logo.png}
    \caption{Logo UPS 2}
    \label{F1}
\end{figure}

%Seccion tablas https://www.tablesgenerator.com/
\section{Tablas}

%Permite centrar la informcion de las celdas, para lineas verticales 
%Permite Gregar lineas c|c|c
\begin{table}[h]
\centering
\begin{tabular}{ccc}
Nombre      & Apellido & Nota \\ \hline \hline
Juan        & Naula    & 38   \\
Karolain    & Naula    & 38   \\
Ian         & Naula    & 38   \\
Jenny       & Quizhpe  & 31   \\ \hline \hline
\end{tabular}
\caption{Notas}
\label{T1}
\end{table}

En la tabla \ref{T1} se muestran las notas de los estudiantes 


\begin{table}[h]
\centering
\begin{tabular}{c|c|c}
\multicolumn{3}{c}{Cuadro de notas} \\ \hline
Nombre     & Apellido     & Nota    \\ 
Juan       & Naula        & 38      \\
Jenny      & Quizhpe      & 31      \\ \hline
\end{tabular}
\end{table}

\section{Como citar} \label{sec:cita}
En la seccion \ref{sec:cita}, se indica como citar en un documento \cite{uno}.

Los autores en \cite{uno} indican que ....

En \cite{uno}, se define ....

La Ingenier�a de sofware es .....,  tal como se indica en \cite{dos}.

En los diferentes estudios \cite{uno,dos}



%\bibliographystyle{IEEEtran}
\bibliographystyle{ieeetr}
\bibliography{bib.bib}

\end{document}